% Options for packages loaded elsewhere
\PassOptionsToPackage{unicode}{hyperref}
\PassOptionsToPackage{hyphens}{url}
%
\documentclass[
  ignorenonframetext,
]{beamer}
\usepackage{pgfpages}
\setbeamertemplate{caption}[numbered]
\setbeamertemplate{caption label separator}{: }
\setbeamercolor{caption name}{fg=normal text.fg}
\beamertemplatenavigationsymbolsempty
% Prevent slide breaks in the middle of a paragraph
\widowpenalties 1 10000
\raggedbottom
\setbeamertemplate{part page}{
  \centering
  \begin{beamercolorbox}[sep=16pt,center]{part title}
    \usebeamerfont{part title}\insertpart\par
  \end{beamercolorbox}
}
\setbeamertemplate{section page}{
  \centering
  \begin{beamercolorbox}[sep=12pt,center]{part title}
    \usebeamerfont{section title}\insertsection\par
  \end{beamercolorbox}
}
\setbeamertemplate{subsection page}{
  \centering
  \begin{beamercolorbox}[sep=8pt,center]{part title}
    \usebeamerfont{subsection title}\insertsubsection\par
  \end{beamercolorbox}
}
\AtBeginPart{
  \frame{\partpage}
}
\AtBeginSection{
  \ifbibliography
  \else
    \frame{\sectionpage}
  \fi
}
\AtBeginSubsection{
  \frame{\subsectionpage}
}
\usepackage{amsmath,amssymb}
\usepackage{iftex}
\ifPDFTeX
  \usepackage[T1]{fontenc}
  \usepackage[utf8]{inputenc}
  \usepackage{textcomp} % provide euro and other symbols
\else % if luatex or xetex
  \usepackage{unicode-math} % this also loads fontspec
  \defaultfontfeatures{Scale=MatchLowercase}
  \defaultfontfeatures[\rmfamily]{Ligatures=TeX,Scale=1}
\fi
\usepackage{lmodern}
\ifPDFTeX\else
  % xetex/luatex font selection
\fi
% Use upquote if available, for straight quotes in verbatim environments
\IfFileExists{upquote.sty}{\usepackage{upquote}}{}
\IfFileExists{microtype.sty}{% use microtype if available
  \usepackage[]{microtype}
  \UseMicrotypeSet[protrusion]{basicmath} % disable protrusion for tt fonts
}{}
\makeatletter
\@ifundefined{KOMAClassName}{% if non-KOMA class
  \IfFileExists{parskip.sty}{%
    \usepackage{parskip}
  }{% else
    \setlength{\parindent}{0pt}
    \setlength{\parskip}{6pt plus 2pt minus 1pt}}
}{% if KOMA class
  \KOMAoptions{parskip=half}}
\makeatother
\usepackage{xcolor}
\newif\ifbibliography
\usepackage{color}
\usepackage{fancyvrb}
\newcommand{\VerbBar}{|}
\newcommand{\VERB}{\Verb[commandchars=\\\{\}]}
\DefineVerbatimEnvironment{Highlighting}{Verbatim}{commandchars=\\\{\}}
% Add ',fontsize=\small' for more characters per line
\usepackage{framed}
\definecolor{shadecolor}{RGB}{248,248,248}
\newenvironment{Shaded}{\begin{snugshade}}{\end{snugshade}}
\newcommand{\AlertTok}[1]{\textcolor[rgb]{0.94,0.16,0.16}{#1}}
\newcommand{\AnnotationTok}[1]{\textcolor[rgb]{0.56,0.35,0.01}{\textbf{\textit{#1}}}}
\newcommand{\AttributeTok}[1]{\textcolor[rgb]{0.13,0.29,0.53}{#1}}
\newcommand{\BaseNTok}[1]{\textcolor[rgb]{0.00,0.00,0.81}{#1}}
\newcommand{\BuiltInTok}[1]{#1}
\newcommand{\CharTok}[1]{\textcolor[rgb]{0.31,0.60,0.02}{#1}}
\newcommand{\CommentTok}[1]{\textcolor[rgb]{0.56,0.35,0.01}{\textit{#1}}}
\newcommand{\CommentVarTok}[1]{\textcolor[rgb]{0.56,0.35,0.01}{\textbf{\textit{#1}}}}
\newcommand{\ConstantTok}[1]{\textcolor[rgb]{0.56,0.35,0.01}{#1}}
\newcommand{\ControlFlowTok}[1]{\textcolor[rgb]{0.13,0.29,0.53}{\textbf{#1}}}
\newcommand{\DataTypeTok}[1]{\textcolor[rgb]{0.13,0.29,0.53}{#1}}
\newcommand{\DecValTok}[1]{\textcolor[rgb]{0.00,0.00,0.81}{#1}}
\newcommand{\DocumentationTok}[1]{\textcolor[rgb]{0.56,0.35,0.01}{\textbf{\textit{#1}}}}
\newcommand{\ErrorTok}[1]{\textcolor[rgb]{0.64,0.00,0.00}{\textbf{#1}}}
\newcommand{\ExtensionTok}[1]{#1}
\newcommand{\FloatTok}[1]{\textcolor[rgb]{0.00,0.00,0.81}{#1}}
\newcommand{\FunctionTok}[1]{\textcolor[rgb]{0.13,0.29,0.53}{\textbf{#1}}}
\newcommand{\ImportTok}[1]{#1}
\newcommand{\InformationTok}[1]{\textcolor[rgb]{0.56,0.35,0.01}{\textbf{\textit{#1}}}}
\newcommand{\KeywordTok}[1]{\textcolor[rgb]{0.13,0.29,0.53}{\textbf{#1}}}
\newcommand{\NormalTok}[1]{#1}
\newcommand{\OperatorTok}[1]{\textcolor[rgb]{0.81,0.36,0.00}{\textbf{#1}}}
\newcommand{\OtherTok}[1]{\textcolor[rgb]{0.56,0.35,0.01}{#1}}
\newcommand{\PreprocessorTok}[1]{\textcolor[rgb]{0.56,0.35,0.01}{\textit{#1}}}
\newcommand{\RegionMarkerTok}[1]{#1}
\newcommand{\SpecialCharTok}[1]{\textcolor[rgb]{0.81,0.36,0.00}{\textbf{#1}}}
\newcommand{\SpecialStringTok}[1]{\textcolor[rgb]{0.31,0.60,0.02}{#1}}
\newcommand{\StringTok}[1]{\textcolor[rgb]{0.31,0.60,0.02}{#1}}
\newcommand{\VariableTok}[1]{\textcolor[rgb]{0.00,0.00,0.00}{#1}}
\newcommand{\VerbatimStringTok}[1]{\textcolor[rgb]{0.31,0.60,0.02}{#1}}
\newcommand{\WarningTok}[1]{\textcolor[rgb]{0.56,0.35,0.01}{\textbf{\textit{#1}}}}
\usepackage{graphicx}
\makeatletter
\def\maxwidth{\ifdim\Gin@nat@width>\linewidth\linewidth\else\Gin@nat@width\fi}
\def\maxheight{\ifdim\Gin@nat@height>\textheight\textheight\else\Gin@nat@height\fi}
\makeatother
% Scale images if necessary, so that they will not overflow the page
% margins by default, and it is still possible to overwrite the defaults
% using explicit options in \includegraphics[width, height, ...]{}
\setkeys{Gin}{width=\maxwidth,height=\maxheight,keepaspectratio}
% Set default figure placement to htbp
\makeatletter
\def\fps@figure{htbp}
\makeatother
\setlength{\emergencystretch}{3em} % prevent overfull lines
\providecommand{\tightlist}{%
  \setlength{\itemsep}{0pt}\setlength{\parskip}{0pt}}
\setcounter{secnumdepth}{-\maxdimen} % remove section numbering
\ifLuaTeX
  \usepackage{selnolig}  % disable illegal ligatures
\fi
\IfFileExists{bookmark.sty}{\usepackage{bookmark}}{\usepackage{hyperref}}
\IfFileExists{xurl.sty}{\usepackage{xurl}}{} % add URL line breaks if available
\urlstyle{same}
\hypersetup{
  pdftitle={Correspondance entre classes étiquettées et Clusters prédits par algorithme hongrois},
  pdfauthor={Willy Kinfoussia, Aurélien Henriques},
  hidelinks,
  pdfcreator={LaTeX via pandoc}}

\title{Correspondance entre classes étiquettées et Clusters prédits par
algorithme hongrois}
\author{Willy Kinfoussia, Aurélien Henriques}
\date{2023-2024 M2DS}

\begin{document}
\frame{\titlepage}

\begin{frame}{}
\protect\hypertarget{section}{}
Introduction

Un problème notable dans l'apprentissage supervisé est la labélisation
d'un grand nombre de données afin d'entraîner un modèle. En effet, Il
arrive souvent que le nombre de données labelisées soit dérisoire
comparément aux données dont l'on dispose au total. Certains
algorithmes, par exemple l'algorithme SWAV, permettent de s'affranchir
de ce problème en créant une représentation de classe par alteration
faible des exemples labélisés, mais ils nécessitent un apprentissage
contrastif afin de pouvoir identifier une classe.

Les algorithmes de clusterisation non-supervisé peuvent alors permettre
de résoudre ce problème, en clusterisant les données par cluster pourvu
que nous puissions identifier correctement les clusters aux étiquettes
de classe. Dans ce cas, il suffirait alors de s'assurer de la bonne
représentation du modèle des différentes classes via les clusters, et
d'apposer aux données des clusters les étiquettes de classe issue de
l'association cluster/classe.
\end{frame}

\begin{frame}{}
\protect\hypertarget{section-1}{}
Dans ce projet, nous proposons d'appliquer l'algorithme hongrois à ce
problème. L'algorithme hongrois est un programme qui permet de trouver
la meilleure association entre N agents et N tâches à l'aide de poids
associés aux arrêtes entre agents et tâches. En pratique, il existe N**2
poids d'arrête et N ! façons différentes d'associer les agents aux
tâches. La recherche de l'optimum des associations peut donc être
pénible et utiliser l'algorithme hongrois permet de résoudre en partie
ce problème, grâce une complexité polynômiale en N. Il repose sur une
recherche du poids d'arrête minimal dans une matrice de coût qui
contiendrait tous les poids des arrêtes entre agents et tâches (dans
notre cas, entre cluster et classe). Nous utilisons ici cet algorithme
ainsi qu'un algorithme naîf parcourant les N! possibilités
\end{frame}

\begin{frame}{}
\protect\hypertarget{section-2}{}
Construction de la matrice de coût

Nous collectons les données MNIST, ainsi que les classes associées pour
obtenir une matrice de coût. Pour ceci, nous effectuons une PCA à 50
dimensions sur les données puis nous clusterisons par k-nn les données
obtenues en 10 clusters (le nombre de classe dans MNIST). Nous calculons
alors les centres d'inertie des classes et des clusters puis l'on
utilise la distance euclidienne entre chaque couple classe/cluster.

\[
\text{d}(c, k) = \sqrt{\sum_{l=1}^{50} (c_{l} - k_{l})^2}
\]

Où : - \(c_l\) représente la valeur du centre d'inertie de la classe
\(c\) en la l-ème dimension, - \(k_l\) représente la valeur du centre
d'inertie du cluster \(k\) en la l-ème dimension,
\end{frame}

\begin{frame}[fragile]{}
\protect\hypertarget{section-3}{}
\begin{Shaded}
\begin{Highlighting}[]
\NormalTok{mnist\_train }\OtherTok{\textless{}{-}} \FunctionTok{read.csv}\NormalTok{(}\FunctionTok{paste0}\NormalTok{(}\StringTok{\textquotesingle{}mnist\_test.csv\textquotesingle{}}\NormalTok{))}

\NormalTok{x\_train }\OtherTok{\textless{}{-}}\NormalTok{ mnist\_train[, }\SpecialCharTok{{-}}\DecValTok{1}\NormalTok{]}
\NormalTok{y\_train }\OtherTok{\textless{}{-}}\NormalTok{ mnist\_train[, }\DecValTok{1}\NormalTok{]}

\NormalTok{x\_train\_vec }\OtherTok{\textless{}{-}}\NormalTok{ x\_train }\SpecialCharTok{/} \DecValTok{255}

\NormalTok{constant\_columns }\OtherTok{\textless{}{-}} \FunctionTok{apply}\NormalTok{(x\_train\_vec, }\DecValTok{2}\NormalTok{, }\ControlFlowTok{function}\NormalTok{(col) }\FunctionTok{length}\NormalTok{(}\FunctionTok{unique}\NormalTok{(col)) }\SpecialCharTok{==} \DecValTok{1}\NormalTok{)}
\NormalTok{x\_train\_vec }\OtherTok{\textless{}{-}}\NormalTok{ x\_train\_vec[, }\SpecialCharTok{!}\NormalTok{constant\_columns]}

\NormalTok{pca }\OtherTok{\textless{}{-}} \FunctionTok{prcomp}\NormalTok{(x\_train\_vec, }\AttributeTok{center =} \ConstantTok{TRUE}\NormalTok{, }\AttributeTok{scale. =} \ConstantTok{TRUE}\NormalTok{, }\AttributeTok{rank =} \DecValTok{50}\NormalTok{)}

\NormalTok{x\_train\_pca }\OtherTok{\textless{}{-}} \FunctionTok{predict}\NormalTok{(pca, x\_train\_vec)}
\end{Highlighting}
\end{Shaded}
\end{frame}

\begin{frame}[fragile]{}
\protect\hypertarget{section-4}{}
\begin{Shaded}
\begin{Highlighting}[]
\FunctionTok{library}\NormalTok{(stats)}

\NormalTok{k }\OtherTok{\textless{}{-}} \DecValTok{10}

\NormalTok{kmeans\_model }\OtherTok{\textless{}{-}} \FunctionTok{kmeans}\NormalTok{(x\_train\_pca, }\AttributeTok{centers =}\NormalTok{ k)}

\NormalTok{clusters\_centroid }\OtherTok{\textless{}{-}}\NormalTok{ kmeans\_model}\SpecialCharTok{$}\NormalTok{centers}

\NormalTok{assignement\_cluster }\OtherTok{\textless{}{-}}\NormalTok{ kmeans\_model}\SpecialCharTok{$}\NormalTok{cluster}

\NormalTok{classes\_centroid }\OtherTok{\textless{}{-}} \FunctionTok{list}\NormalTok{()}
\end{Highlighting}
\end{Shaded}
\end{frame}

\begin{frame}{}
\protect\hypertarget{section-5}{}
Optimisation

La matrice de coût est alors :

\[
\text{M} = (d_{ck}) 
\]

Où : - c et k varient entre 1 et 10.

La matrice \(X\) d'association classe/cluster de taille \((10,10)\) est
définie comme :

\[
X_{ck} = 
\begin{cases} 
1 & \text{si la classe } c \text{ et le cluster } k \text{ sont associés} \\
0 & \text{sinon} 
\end{cases}
\]

avec la contrainte qu'il ne doit y avoir qu'un seul \(1\) par ligne et
des \(0\) sur le reste, correspondant à ne pas associer la classe c aux
autres clusters.

\[
\ \sum_{k=1}^{10} X_{ck} = 1
\]
\end{frame}

\begin{frame}[fragile]{}
\protect\hypertarget{section-6}{}
\begin{Shaded}
\begin{Highlighting}[]
\ControlFlowTok{for}\NormalTok{ (class\_label }\ControlFlowTok{in} \DecValTok{0}\SpecialCharTok{:}\NormalTok{(k }\SpecialCharTok{{-}} \DecValTok{1}\NormalTok{)) \{}
  
\NormalTok{  class\_data }\OtherTok{\textless{}{-}}\NormalTok{ x\_train\_pca[y\_train }\SpecialCharTok{==}\NormalTok{ class\_label, ]}
  
\NormalTok{  centroid }\OtherTok{\textless{}{-}} \FunctionTok{colMeans}\NormalTok{(class\_data)}
  
\NormalTok{  classes\_centroid[[}\FunctionTok{as.character}\NormalTok{(class\_label)]] }\OtherTok{\textless{}{-}}\NormalTok{ centroid}
\NormalTok{\}}

\NormalTok{cost\_matrix\_mnist }\OtherTok{\textless{}{-}} \FunctionTok{matrix}\NormalTok{(}\DecValTok{0}\NormalTok{, }\AttributeTok{nrow =}\NormalTok{ k, }\AttributeTok{ncol =}\NormalTok{ k)}

\ControlFlowTok{for}\NormalTok{ (i }\ControlFlowTok{in} \DecValTok{0}\SpecialCharTok{:}\NormalTok{k}\DecValTok{{-}1}\NormalTok{) \{}
  \ControlFlowTok{for}\NormalTok{ (j }\ControlFlowTok{in} \DecValTok{1}\SpecialCharTok{:}\NormalTok{k) \{}

\NormalTok{    distance }\OtherTok{\textless{}{-}} \FunctionTok{sqrt}\NormalTok{(}\FunctionTok{sum}\NormalTok{((classes\_centroid[[}\FunctionTok{as.character}\NormalTok{(i)]] }\SpecialCharTok{{-}}\NormalTok{ clusters\_centroid[j, ])}\SpecialCharTok{\^{}}\DecValTok{2}\NormalTok{))}
    
\NormalTok{    cost\_matrix\_mnist[i}\SpecialCharTok{+}\DecValTok{1}\NormalTok{, j] }\OtherTok{\textless{}{-}}\NormalTok{ distance}
\NormalTok{  \}}
\NormalTok{\}}

\NormalTok{cost\_matrix\_mnist }\OtherTok{\textless{}{-}} \FunctionTok{lapply}\NormalTok{(}\DecValTok{1}\SpecialCharTok{:}\NormalTok{k, }\ControlFlowTok{function}\NormalTok{(i) }\FunctionTok{as.vector}\NormalTok{(cost\_matrix\_mnist[i, ]))}
\end{Highlighting}
\end{Shaded}
\end{frame}

\begin{frame}[fragile]{}
\protect\hypertarget{section-7}{}
\begin{Shaded}
\begin{Highlighting}[]
\CommentTok{\#Génération de matrice de coût aléatoire}

\NormalTok{matrice\_couts }\OtherTok{\textless{}{-}} \ControlFlowTok{function}\NormalTok{(n) \{}
  
\NormalTok{  cost\_matrix\_random }\OtherTok{\textless{}{-}} \FunctionTok{matrix}\NormalTok{(}\FunctionTok{rnorm}\NormalTok{(n }\SpecialCharTok{*}\NormalTok{ n, }\AttributeTok{mean =} \DecValTok{100}\NormalTok{, }\AttributeTok{sd =} \DecValTok{15}\NormalTok{), }\AttributeTok{nrow =}\NormalTok{ n, }\AttributeTok{ncol =}\NormalTok{ n)}
  \CommentTok{\#cost\_matrix\_random \textless{}{-} pnorm(matrice\_gaussienne)}
\NormalTok{  liste\_matrice }\OtherTok{\textless{}{-}} \FunctionTok{lapply}\NormalTok{(}\DecValTok{1}\SpecialCharTok{:}\NormalTok{n, }\ControlFlowTok{function}\NormalTok{(i) }\FunctionTok{as.vector}\NormalTok{(cost\_matrix\_random[i, ]))}
  
  \FunctionTok{return}\NormalTok{(liste\_matrice)}
\NormalTok{\}}
\end{Highlighting}
\end{Shaded}
\end{frame}

\begin{frame}{}
\protect\hypertarget{section-8}{}
Comme on souhaite aussi qu'un cluster corresponde à une ligne (
éventuellement il est possible de faire concorder plusieurs clusters à
une classe et inversement en enlevant les contraintes sur la bijection
de notre association), une seconde contrainte est qu'il ne doit y avoir
qu'un 1 par ligne et des 0 sur le reste.

\[
\ \sum_{c=1}^{10} X_{ck} = 1
\]

La fonction de coût à minimiser est la somme des produits
\(X_{kl} \times M_{kl}\) sur \(l\) et \(k\) :

\[
\text{Coût(association X)} = \sum_{c=1}^{10} \sum_{k=1}^{10} X_{ck} \times M_{ck}
\]

Cette fonction correspond au coût de l'association classe/cluster
choisie. En la minimisant, on s'assure que l'association permet au
maximum de réduire la distance séparant le cluster de la classe.
\end{frame}

\begin{frame}{}
\protect\hypertarget{section-9}{}
Algorithme naîf

L'algorithme naîf permet de résoudre ce problème brutalement en
calculant toutes les coûts possibles. C'est-à-dire calculer pour chaque
bijection classes-cluster la somme des arrêtes associant un cluster à
une classe et renvoyer la bijection qui minimise le coût. La complexité
théorique de cet algorithmique est N! où N est le nombre de
classes/clusters. En effet, pour la classe 1, il y a 10 clusters
possibles. Une fois un cluster choisi, on obtient le premier coefficient
de coût. Il reste donc pour la classe 2 9 clusters possibles, 8 pour la
classe 3, etc. Ceci donne bien N! associations possibles donc N! somme
des coûts possibles, ce qui donne la complexité de cet algorithme.
\end{frame}

\begin{frame}[fragile]{}
\protect\hypertarget{section-10}{}
Complexité expérimentale

\begin{Shaded}
\begin{Highlighting}[]
\NormalTok{temps\_execution }\OtherTok{\textless{}{-}} \ControlFlowTok{function}\NormalTok{(n\_liste,p,algo) \{}

\NormalTok{  temps }\OtherTok{\textless{}{-}} \FunctionTok{numeric}\NormalTok{(}\FunctionTok{length}\NormalTok{(n\_liste))}
  
  \ControlFlowTok{for}\NormalTok{ (j }\ControlFlowTok{in} \FunctionTok{seq\_along}\NormalTok{(n\_liste)) \{}
\NormalTok{    n }\OtherTok{\textless{}{-}}\NormalTok{ n\_liste[j]}
\NormalTok{    cout }\OtherTok{\textless{}{-}} \FunctionTok{matrice\_couts}\NormalTok{(n)}
    
\NormalTok{    start\_time }\OtherTok{\textless{}{-}} \FunctionTok{Sys.time}\NormalTok{()}
    
    \ControlFlowTok{for}\NormalTok{ (i }\ControlFlowTok{in} \DecValTok{0}\SpecialCharTok{:}\NormalTok{(p}\DecValTok{{-}1}\NormalTok{)) \{}
    \FunctionTok{algo}\NormalTok{(cout)}
\NormalTok{    \}}
    
\NormalTok{    end\_time }\OtherTok{\textless{}{-}} \FunctionTok{Sys.time}\NormalTok{()}
\NormalTok{    temps[j] }\OtherTok{\textless{}{-}}\NormalTok{ (end\_time }\SpecialCharTok{{-}}\NormalTok{ start\_time)}\SpecialCharTok{/}\NormalTok{p}
\NormalTok{  \}}
  \FunctionTok{return}\NormalTok{(temps)}
\NormalTok{\}}
\end{Highlighting}
\end{Shaded}
\end{frame}

\begin{frame}[fragile]{}
\protect\hypertarget{section-11}{}
\begin{Shaded}
\begin{Highlighting}[]
\FunctionTok{library}\NormalTok{(Rcpp)}
\end{Highlighting}
\end{Shaded}

\begin{verbatim}
## Warning: le package 'Rcpp' a été compilé avec la version R 4.3.3
\end{verbatim}

\begin{Shaded}
\begin{Highlighting}[]
\FunctionTok{sourceCpp}\NormalTok{(}\StringTok{"NaiveAlgorithme.cpp"}\NormalTok{)}
\end{Highlighting}
\end{Shaded}

\begin{Shaded}
\begin{Highlighting}[]
\NormalTok{liste\_dimension }\OtherTok{\textless{}{-}} \FunctionTok{c}\NormalTok{(}\DecValTok{4}\NormalTok{,}\DecValTok{5}\NormalTok{,}\DecValTok{6}\NormalTok{,}\DecValTok{7}\NormalTok{,}\DecValTok{8}\NormalTok{,}\DecValTok{9}\NormalTok{,}\DecValTok{10}\NormalTok{)}

\NormalTok{temps\_exe }\OtherTok{\textless{}{-}} \FunctionTok{log}\NormalTok{(}\FunctionTok{temps\_execution}\NormalTok{(liste\_dimension,}\DecValTok{5}\NormalTok{,NaiveAlgorithme)) }

\FunctionTok{plot}\NormalTok{(}\FunctionTok{log}\NormalTok{(liste\_dimension),temps\_exe,}\AttributeTok{xlab=}\StringTok{\textquotesingle{}log(nombre de classes)\textquotesingle{}}\NormalTok{,}\AttributeTok{ylab=}\StringTok{\textquotesingle{}log(temps\textquotesingle{}}\NormalTok{)}
\end{Highlighting}
\end{Shaded}

\includegraphics{main_copie_files/figure-beamer/unnamed-chunk-7-1.pdf}
\end{frame}

\begin{frame}[fragile]{}
\protect\hypertarget{section-12}{}
Comme la complexité théorique est en N! on déduit une régression de \[
\ log(T)=f(N*log(N)-N)
\]

avec

\[
\ N*(log(N)-N) ≈  log(N!)
\]

\begin{Shaded}
\begin{Highlighting}[]
\NormalTok{logn\_n }\OtherTok{\textless{}{-}}\NormalTok{(}\FunctionTok{log}\NormalTok{(liste\_dimension)}\SpecialCharTok{{-}}\DecValTok{1}\NormalTok{)}\SpecialCharTok{*}\NormalTok{liste\_dimension}

\NormalTok{reg }\OtherTok{\textless{}{-}} \FunctionTok{lm}\NormalTok{(temps\_exe}\SpecialCharTok{\textasciitilde{}}\NormalTok{logn\_n)}

\FunctionTok{summary}\NormalTok{(reg)}
\end{Highlighting}
\end{Shaded}

\begin{verbatim}
## 
## Call:
## lm(formula = temps_exe ~ logn_n)
## 
## Residuals:
##       1       2       3       4       5       6       7 
##  3.0703 -1.1823 -1.7172 -1.1298 -0.7040  0.4823  1.1807 
## 
## Coefficients:
##             Estimate Std. Error t value Pr(>|t|)    
## (Intercept) -12.2612     1.4398  -8.516 0.000367 ***
## logn_n        0.6402     0.1819   3.519 0.016937 *  
## ---
## Signif. codes:  0 '***' 0.001 '**' 0.01 '*' 0.05 '.' 0.1 ' ' 1
## 
## Residual standard error: 1.853 on 5 degrees of freedom
## Multiple R-squared:  0.7124, Adjusted R-squared:  0.6548 
## F-statistic: 12.38 on 1 and 5 DF,  p-value: 0.01694
\end{verbatim}

\begin{Shaded}
\begin{Highlighting}[]
\FunctionTok{print}\NormalTok{(}\StringTok{"complexité :"}\NormalTok{)}
\end{Highlighting}
\end{Shaded}

\begin{verbatim}
## [1] "complexité :"
\end{verbatim}

\begin{Shaded}
\begin{Highlighting}[]
\FunctionTok{print}\NormalTok{(}\FunctionTok{coef}\NormalTok{(reg)[}\DecValTok{2}\NormalTok{])}
\end{Highlighting}
\end{Shaded}

\begin{verbatim}
##    logn_n 
## 0.6402042
\end{verbatim}
\end{frame}

\begin{frame}{}
\protect\hypertarget{section-13}{}
Algorithme hongrois

Étape 0 : Soustraire le minimum de chaque ligne à tous les éléments de
cette ligne, puis soustraire le minimum de chaque colonne à tous les
éléments de cette colonne dans la matrice de coût.

Étape 1 : Sélectionner le maximum de zéros indépendants, c'est-à-dire un
seul zéro par ligne et par colonne, en parcourant tous les zéros non
sélectionnés et en les sélectionnant s'ils ne partagent pas la même
ligne ou colonne qu'un zéro déjà sélectionné.

Étape 2 : Couvrir chaque colonne ayant un zéro sélectionné. Puis, pour
chaque zéro non couvert, marquer la colonne de ce zéro et couvrir la
ligne de ce zéro si aucune colonne n'est marquée pour ce zéro. Répéter
jusqu'à ce qu'il n'y ait plus de zéros non couverts.

Étape 3 : Trouver la valeur minimum des éléments non couverts dans la
matrice. Ajouter cette valeur à toutes les lignes couvertes et la
retirer à toutes les colonnes non couvertes. Recommencer à l'étape 1
\end{frame}

\begin{frame}{}
\protect\hypertarget{section-14}{}
Complexité

Les différentes étapes sont faites avec une complexité de 1 sur la
totalité de la matrice de coût donc ont une complexité maximale de
O(n\^{}2). De plus, Une itération de ces étapes garantit d'avoir au
moins un zéro indépendant, ce qui indique que l'algorithme résout le
problème au maximum en n\^{}2 itérations, correspondant au cas le cas le
plus extrême, à devoir itérer sur chacun des coefficients de la matrice
d'assignation, qui a n\^{}2 coefficients. On en déduit que l'algorithme
a une complexité théorique en O(n\^{}4)
\end{frame}

\begin{frame}[fragile]{}
\protect\hypertarget{section-15}{}
\begin{Shaded}
\begin{Highlighting}[]
\FunctionTok{sourceCpp}\NormalTok{(}\StringTok{"Hungarian.cpp"}\NormalTok{)}
\end{Highlighting}
\end{Shaded}

\begin{Shaded}
\begin{Highlighting}[]
\NormalTok{temps\_exe\_hong }\OtherTok{\textless{}{-}} \FunctionTok{log}\NormalTok{(}\FunctionTok{temps\_execution}\NormalTok{(liste\_dimension,}\DecValTok{5}\NormalTok{,Hungarian))}

\FunctionTok{plot}\NormalTok{(}\FunctionTok{log}\NormalTok{(liste\_dimension),temps\_exe\_hong,}\AttributeTok{xlab=}\StringTok{\textquotesingle{}log(nombre de classes)\textquotesingle{}}\NormalTok{,}\AttributeTok{ylab=}\StringTok{\textquotesingle{}log(temps\textquotesingle{}}\NormalTok{)}
\end{Highlighting}
\end{Shaded}

\includegraphics{main_copie_files/figure-beamer/unnamed-chunk-10-1.pdf}

\begin{Shaded}
\begin{Highlighting}[]
\NormalTok{reg }\OtherTok{\textless{}{-}} \FunctionTok{lm}\NormalTok{(temps\_exe\_hong}\SpecialCharTok{\textasciitilde{}}\FunctionTok{log}\NormalTok{(liste\_dimension))}

\FunctionTok{summary}\NormalTok{(reg)}
\end{Highlighting}
\end{Shaded}

\begin{verbatim}
## 
## Call:
## lm(formula = temps_exe_hong ~ log(liste_dimension))
## 
## Residuals:
##        1        2        3        4        5        6        7 
##  1.64310 -1.29615 -0.94051 -0.36517 -0.02680  0.09044  0.89508 
## 
## Coefficients:
##                      Estimate Std. Error t value Pr(>|t|)  
## (Intercept)            -6.891      2.672  -2.578   0.0495 *
## log(liste_dimension)   -2.054      1.388  -1.480   0.1989  
## ---
## Signif. codes:  0 '***' 0.001 '**' 0.01 '*' 0.05 '.' 0.1 ' ' 1
## 
## Residual standard error: 1.114 on 5 degrees of freedom
## Multiple R-squared:  0.3047, Adjusted R-squared:  0.1656 
## F-statistic: 2.191 on 1 and 5 DF,  p-value: 0.1989
\end{verbatim}

\begin{Shaded}
\begin{Highlighting}[]
\FunctionTok{print}\NormalTok{(}\StringTok{"complexité polynomiale :"}\NormalTok{)}
\end{Highlighting}
\end{Shaded}

\begin{verbatim}
## [1] "complexité polynomiale :"
\end{verbatim}

\begin{Shaded}
\begin{Highlighting}[]
\FunctionTok{print}\NormalTok{(}\FunctionTok{coef}\NormalTok{(reg)[}\DecValTok{2}\NormalTok{])}
\end{Highlighting}
\end{Shaded}

\begin{verbatim}
## log(liste_dimension) 
##            -2.054124
\end{verbatim}
\end{frame}

\begin{frame}[fragile]{}
\protect\hypertarget{section-16}{}
Performances sur MNIST

On utilise le s clusters associés aux classes pour prédire le test de
mnist. Pour ce faire, nous utilisons la matrice d'assignation
classe/cluster obtenue pour assigner les classes aux clusters prédits
par l'algorithme des k-means. Nous comparons ensuite avec la vraie
classe des données test mnist.

\begin{Shaded}
\begin{Highlighting}[]
\CommentTok{\#clusters\_test \textless{}{-} predict(kmeans\_model, newdata = x\_test\_pca)}

\CommentTok{\# association cluster/classe par Hungarian(cost\_matrix\_mnist)}
\end{Highlighting}
\end{Shaded}
\end{frame}

\begin{frame}[fragile]{}
\protect\hypertarget{section-17}{}
\begin{Shaded}
\begin{Highlighting}[]
\NormalTok{cost\_matrix }\OtherTok{\textless{}{-}} \FunctionTok{list}\NormalTok{(}
  \FunctionTok{c}\NormalTok{(}\DecValTok{8}\NormalTok{, }\DecValTok{6}\NormalTok{, }\DecValTok{5}\NormalTok{),}
  \FunctionTok{c}\NormalTok{(}\DecValTok{5}\NormalTok{, }\DecValTok{4}\NormalTok{, }\DecValTok{7}\NormalTok{),}
  \FunctionTok{c}\NormalTok{(}\DecValTok{8}\NormalTok{, }\DecValTok{4}\NormalTok{, }\DecValTok{6}\NormalTok{)}
\NormalTok{)}
\NormalTok{naive\_adj\_matrix }\OtherTok{\textless{}{-}} \FunctionTok{NaiveAlgorithme}\NormalTok{(cost\_matrix)}
\NormalTok{naive\_adj\_matrix}
\end{Highlighting}
\end{Shaded}

\begin{verbatim}
## [[1]]
## [1] 0 0 1
## 
## [[2]]
## [1] 1 0 0
## 
## [[3]]
## [1] 0 1 0
\end{verbatim}

\begin{Shaded}
\begin{Highlighting}[]
\NormalTok{cost\_matrix }\OtherTok{\textless{}{-}} \FunctionTok{list}\NormalTok{(}
  \FunctionTok{c}\NormalTok{(}\DecValTok{8}\NormalTok{, }\DecValTok{6}\NormalTok{, }\DecValTok{5}\NormalTok{),}
  \FunctionTok{c}\NormalTok{(}\DecValTok{5}\NormalTok{, }\DecValTok{4}\NormalTok{, }\DecValTok{7}\NormalTok{),}
  \FunctionTok{c}\NormalTok{(}\DecValTok{8}\NormalTok{, }\DecValTok{4}\NormalTok{, }\DecValTok{6}\NormalTok{)}
\NormalTok{)}
\NormalTok{hungarian\_adj\_matrix }\OtherTok{\textless{}{-}} \FunctionTok{Hungarian}\NormalTok{(cost\_matrix)}
\NormalTok{hungarian\_adj\_matrix}
\end{Highlighting}
\end{Shaded}

\begin{verbatim}
## [[1]]
## [1] 0 0 1
## 
## [[2]]
## [1] 1 0 0
## 
## [[3]]
## [1] 0 1 0
\end{verbatim}
\end{frame}

\end{document}
